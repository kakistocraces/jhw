% Created 2021-01-22 Fri 07:59
% Intended LaTeX compiler: pdflatex
\documentclass[11pt]{article}
\usepackage[utf8]{inputenc}
\usepackage[T1]{fontenc}
\usepackage{graphicx}
\usepackage{grffile}
\usepackage{longtable}
\usepackage{wrapfig}
\usepackage{rotating}
\usepackage[normalem]{ulem}
\usepackage{amsmath}
\usepackage{textcomp}
\usepackage{amssymb}
\usepackage{capt-of}
\usepackage{hyperref}
\usepackage[english]{babel}
\author{Nikhil Dhawan, MD}
\date{01-21-21}
\title{Org Mode and Homework}
\hypersetup{
 pdfauthor={Nikhil Dhawan, MD},
 pdftitle={Org Mode and Homework},
 pdfkeywords={},
 pdfsubject={},
 pdfcreator={Emacs 27.1 (Org mode 9.3)}, 
 pdflang={English}}
\begin{document}

\maketitle
\tableofcontents

\begin{verbatim}
import os
import csv
\end{verbatim}

\begin{verbatim}



\end{verbatim}


Here I am declaring dictionary lists to store the total votes and the percentages. Each candidate is the key for the dict file. The two dictionary lists are \texttt{totals} and \texttt{percentages} which will contain the total votes and the percentage of votes received for each candidate. 

\begin{verbatim}
totals = {}
percentages = {}
\end{verbatim}

\begin{verbatim}



\end{verbatim}


For the four candidates Khan, O'Tooley, Li, and Correy, I am setting their total votes to 0. By setting the totals to 0, I am also declaring the variables. If you try to run the line \texttt{\texttt{totals["Khan"] = totals["Khan"] + 1}} you will get an error, unless the key has been initiated. 

\begin{verbatim}

totals["Khan"] = 0
totals["O'Tooley"] = 0
totals["Li"] = 0
totals["Correy"] = 0
total_vote = 0

\end{verbatim}

\begin{verbatim}



\end{verbatim}


The following code opens election data csv file. Then each row is inputed into \texttt{csv\_reader} which is a \texttt{csv\_reader} object. You can think of it as an array of lists. Each list will contain a list of values that were separated by a comma. 

\begin{verbatim}
csvpath = os.path.join("/home/nikd/Dropbox/jhw","Resources", "election_data.csv")  

with open(csvpath) as csvfile:
    csv_reader = csv.reader(csvfile, delimiter = ",")
    csv_header = next(csv_reader)
\end{verbatim}

\begin{verbatim}



\end{verbatim}


You cannot iterate through the \texttt{csv\_reader} object with brackets "\texttt{[]}". With an array, you could get the first value of the array with the command \texttt{array\_variable[0]}. This does not work with \texttt{csv\_reader}. Instead you need to iterate through the values with a \texttt{for} loop. The command for this is 
\texttt{for <variable name> in csv\_reader}. In this example, the variable name is \texttt{row}. 

\begin{verbatim}
    for row in csv_reader:
	total_vote = total_vote + 1
	candidate = row[2]
	totals[candidate] += 1

    for candidate in totals:
	percentages[candidate] = totals[candidate] / total_vote * 100

    for candidate in totals:
	print(candidate, " Total = ", totals[candidate], " Percentage = ", percentages[candidate], "%")

print("Winner = ", max(totals, key=totals.get))
\end{verbatim}
\end{document}